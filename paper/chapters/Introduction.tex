\section*{Introduction}

This paper focusses on describing the different steps to reconstruct the path of an excursion through a museum. The first section describes how the paintings are cut out of high-quality pictures. The following section explains an algorithm that can independently find paintings in a picture. Prior research has been done on this topic, one of which is by He et al. \cite{he2019scan}. This paper focusses on the detection of irregularly shaped objects in images and was used as inspiration for the automatic mask creation in the detection algorithm further described in this document. The third section is about how paintings are being matched which was inspired by the work of Liu et al. \cite{liu2020image}. This section describes performant solutions to figure out which paintings are visible in a picture. The fourth section is about how to localize where a person is. This is done by using a custom implementation of the Hidden Markov model to decide which location is a logical result. This model was designed by consolidating the concepts of a paper about a basic Hidden Markov model \cite{eddy1996hidden}, one about speech recognition \cite{juang1991hidden} and a paper about combining historic probabilities \cite{genest1986combining}. The fifth and final section is about the visualization of the result. This section describes a modern way of visualizing the visited rooms in the museum.

\section{Semi-supervised painting detection}
\label{sec:assignment1}

\subsection{The naive painting detection algorithm}
\label{subsec:contour_detection}

The contour detection is the core part of the semi-supervised painting detection. It's the part that decides what contours are in the image and it works as described below.

Before any contour detection can be done, as many unnecessary details as possible have to be removed. OpenCV has several options to solve this problem, including eroding, dilating, blurring (median, Gaussian\dots) or downscaling. The algorithm needs to get rid of as many details as possible to prevent incorrect detection. Although this won't be perfect, the removal of details will decrease the number of incorrectly detected contours.

The first step of the algorithm is to remove details by resizing the image. The current implementation scales the image down and back up with a factor 5. To scale the image back up, pixel values are calculated using a pixel's area. The scale-up was mostly done to be able to show the original image, but it's also possible to use the algorithm with the downscaled image. This will result in a slight increase in performance because the image it's working on would be smaller.

The next step is to convert the image to grayscale. This makes it easier to differentiate certain parts. This grayscaled image is then dilated and eroded to remove even more noise at the borders of the painting or on the walls. The last step removes noise by applying a median blur. The biggest advantage of using a median blur is that it will preserve edges while removing noise. The idea behind these steps is to smear as many details as possible, in other words, to make bigger blots with the same color. This makes it easier to detect the borders and remove noise in the background.

The next step is to detect edges with Canny. The result of the Canny function will then be dilated once again to make the found edges stronger. Thereafter the contours can be detected using OpenCV's contour detection algorithm. This algorithm will make sure only the most outer contours are returned when a hierarchical structure of contours is found. Finally, to prevent illogical solutions, any contour with a ratio smaller than 1:10 will be removed. This prevents very small contours to appear around noisy parts of the image.

\subsection{Strengths and weaknesses}

This biggest strength of this algorithm is that it will give an output in almost every image. An example of an image where no painting will be detected is an image where the framework of the painting is invisible, meaning the image only contains the painting itself, no wall and no framework. With this kind of painting the desired solution is a contour containing the entire painting, while this is impossible because the painting has no border at all. Another advantage is that the algorithm is very fast in detecting contours due to the fact that all parts of the algorithm are standard functions that have a good performance.

However, the biggest weakness of this algorithm is that almost every found contour is not precise enough, so almost every solution needs a slight adjustment. For some solutions, the algorithm detects contours that are slightly larger than the actual painting, while for other paintings, it doesn't even include the border of the painting.

Another weakness of the algorithm is that paintings sometimes have overexposure or shadows, which makes it harder to detect contours. Some paintings even have a small tag with a description next to the painting, sometimes this small tag is detected as being a painting. This is a logical decision because there's a big contrast in colors and the tag has a clear contour, but this is not the desired effect.

This algorithm is a first version to quickly be able to fill the database with the ground truth. The actual algorithm in its final version is completely different and doesn't have these issues anymore. This algorithm will be discussed in the next section.
